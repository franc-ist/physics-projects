\documentclass[twocolumn,prl,nobalancelastpage,aps,10pt]{revtex4-1}
%\documentclass[rmp,preprint]{revtex4-1}
\usepackage{graphicx,bm,times, amsmath}

\usepackage{float} %useful to structure the figures if they're not going where you'd like them to. When beginning a figure environment with "\begin{figure}" add An H to the end to make the placement more forceful like "\begin{figure}[H]"

\usepackage{gensymb} %has some extra symbols like the degree symbol 

\graphicspath{{figures/}}

\begin{document}

\title{Investigating the motion of a rocket in orbit}

\author{Francis Taylor}

\affiliation{Level 5 Laboratory, School of Physics, University of Bristol.}

\date{\today}

\begin{abstract}
The motion of a satellite in orbit was modelled using Newton's Universal Law of Gravitation and solved using the 4th-order Runge-Kutta method for solving ordinary differential equations. The effect on varying the initial height and velocity of the satellite were investigated, and it was found that very small changes in these values could greatly affect the path taken. This model was expanded to simulate two paths of a rocket (figure-of-eight and elliptical) from Low Earth Orbit to the Moon, and back. The figure-of-eight path was found to have a shorter period (9 days, 20 hours and 6 minutes) than the elliptical path (12 days, 3 hours and 36 minutes) and was more stable, with a wider tolerance in the initial values that could give a successful path.
\end{abstract}


\maketitle

\section{INTRODUCTION AND THEORY}

A satellite in orbit around a body follows an elliptical path, with the barycenter at one of the foci of the ellipse, as described by Kepler's First Law \cite{Kepler1992}. This path is defined by the changing velocity of the satellite as it orbits, being lowest when it is at the apoapsis,  which is derived from Newton's laws of motion and gravitation. The force on two bodies with mass is given by the latter, being \cite{Newton1848}

\begin{equation}\label{universalGravitation}
	\bm{F_{12}} = - \frac{G m_{1}m_{2}}{\mid\bm{r_{12}}\mid^{2}}\bm{\hat{r}} ,
\end{equation}

where $\bm{F_{12}}$ is the force between objects 1 and 2, $G$ is the gravitational constant, $\bm{r_{12}}$ is the distance between the center of the objects (see figure \ref{gravitationFigure}), and $m_{1}$ and $m_{2}$ are the masses of the objects.

\begin{figure}[h!]
\includegraphics*[width=0.96\linewidth,clip]{newtonGravitation}
\caption{Newton's Law of Universal Gravitation} \label{gravitationFigure}
\end{figure}

From this, the equation of motion for an orbiting satellite can be derived using Newton's second law, giving \cite{Newton1848}

\begin{equation}\label{equationMotion}
m \bm{\ddot{r}} = - \frac{mMG}{\mid\bm{r}\mid^{3}}\bm{r} ,
\end{equation}

where $m$ is the mass of the satellite, $M$ is the mass of the large body, and $r$ is the position of the satellite relative to the centre of the large body. This is an ordinary differential equation, which can be solved using numerical analysis. One solution of this is using the Runge-Kutta family of iterative methods.

\subsection{Runge-Kutta Methods}

The Runge-Kutta methods are used to provide approximate solutions to ordinary differential equations. Such methods discretise a continuous function in both space and time, allowing for integration over discrete time intervals, a technique easily done using computer simulations \cite{DeVries2011}. For this simulation, the 4th-order Runge-Kutta (RK4) method was used. This method evaluates the slope of a function with known initial values at four different points in a given interval, shown in figure \ref{rk4SlopesFigure}, given by the following \cite{Press2007, Atkinson1989}


\begin{align}
k_{1}& = f(x_{n}, y_{n}) \\
k_{2}& = f(x_{n} + \frac{h}{2}, y_{n} + \frac{hk_{1}}{2})\\
k_{3}& = f(x_{n} + \frac{h}{2}, y_{n} + \frac{hk_{2}}{2})\\
k_{4}& = f(x_{n} + h, y_{n} + hk_{3})
\end{align}

\begin{figure}[ht]
	\includegraphics*[width=0.96\linewidth,clip]{rk4Slopes}
	\caption{Slopes used by the 4th-order Runge-Kutta method.} \label{rk4SlopesFigure}
\end{figure}

In the context of solving equation \ref{equationMotion}, $h$ is the time step, and $x$ and $y$ are the positional coordinates of the rocket. $k_{1}$ is the slope at the beginning of the interval, calculated using $y$. $k_{2}$ is the slope at the midpoint of the interval, calculated using $y$ and $k_{1}$. $k_{2}$ is also the slope at the midpoint of the interval, but is instead calculated using $k_{2}$ instead of $k_{1}$. Finally, $k_{4}$ is the slope at the end of the interval, calculated using $y$ and $k_{4}$.

The solution to this is then given by the weighted average of the increments \cite{Press2007}

\begin{align}
y_{n+1}& = y_{n} + \frac{h}{6}(k_{1} + 2k_{2} + 2k_{3} + k_{4}) , \\
t_{n+1}& = t_{n} + h .
\end{align}\label{rungeEquation}

As this method is of 4th-order, the global truncation error is of  4th-order, so this method is much more accurate than comparable methods, such as Euler's method, which is 2nd-order \cite{Atkinson1989, Suli2003}. Also of note is that if $f$ is independent of $y$, then RK4 simplifies to Simpson's rule \cite{Suli2003}.


A two-dimensional orbit with the larger body at the origin requires several variables that need to be evaluated in four functions, as RK4 couples the variables together:

\begin{align}
f_{1}(t, xx, y, v_{x}, v_{y})& = \frac{dx}{dt} = v_{x}\\
f_{2}(t, xx, y, v_{x}, v_{y})& = \frac{dy}{dt} = v_{y}\\
f_{3}(t, xx, y, v_{x}, v_{y})& = \frac{dv_{x}}{dt} = \frac{-GMx}{(x^{2} + y^{2})^{3/2}}\\
f_{4}(t, xx, y, v_{x}, v_{y})& = \frac{dv_{y}}{dt} = \frac{-GMy}{(x^{2} + y^{2})^{3/2}}
\end{align}

%NOTE THIS MAY NEED TO BE UPDATED
Ignoring the arguments that are not needed, we can apply these functions to each of equations 3-6 to obtain values for $x$, $y$, $v_{x}$ and $v_{y}$ at each increment in the interval. Using this, and applying it to equation 7, a series of time-stepping equations is obtained, where $y$ is replaced by each of the variables listed. These can easily be calculated using a computer, by iterating through a loop, with each increment in the interval stored in an array.

\section{RESULTS AND DISCUSSION}

A circular Medium Earth orbit was modelled for 60000s with a time step of 2s, an initial height of 3621km above the Earth's surface and initial velocity of 6313ms$^{-1}$, shown in figure \ref{circDefFig}. The orbit was stable, completing several revolutions in this time period, and this is also confirmed by the energy graph shown in \ref{circDefEn}, as the energy remains constant over time. It can be inferred from this that the rocket remained on an gravitational equipotential line, hence it's height above the surface is constant.

\begin{figure}[ht!]
\includegraphics*[width=0.96\linewidth,clip]{circularDefault}
\caption{Position of the orbit of a rocket with initial height of 3621km above the Earth's surface and velocity 6313ms$^{-1}$. The path of the rocket is shown in red, the red dot represents the current position of the rocket, and the Earth is represented by the green circle.} \label{circDefFig}
\end{figure}

\begin{figure}[ht]
	\includegraphics*[width=0.96\linewidth,clip]{circularDefaultEnergy}
	\caption{A plot of energy against time for the circular orbit shown in figure \ref{circDefFig}. Gravitational energy is shown by the blue line, kinetic energy by the orange line, and total energy by the green line in between.} \label{circDefEn}
\end{figure}

If the initial height or velocity was too high, the rocket would not enter into a stable orbit, instead moving along a straight path into space, as shown in figure \ref{circHighFig}, or crashing into the Earth. Here, the initial height was increased by 5km to 8621km above the surface of the Earth (15000km from the centre of the Earth), but the initial velocity was kept equal to that in figure \ref{circDefFig}. The rocket was deemed to have crashed if the magnitude of  $x$ and $y$ was less than the radius of the Earth. When simulating the Moon orbit, the same condition was used for the Earth, but a check added to ensure the y-coordinate of the rocket was less than half the distance between the surfaces of the Earth and Moon, $r$. For crashing into the Moon, the code checked if the y-coordinate of the rocket was greater than half the distance between the surfaces, and the magnitude of $x$ and $y-r$ was less than the radius of the Moon.

\begin{figure}[ht]
	\includegraphics*[width=0.96\linewidth,clip]{circularTooHigh}
	\caption{Position of the orbit of a rocket with initial height of 8621km above the Earth's surface and velocity 6313ms$^{-1}$. The path of the rocket is shown in red, the red dot represents the current position of the rocket, and the Earth is represented by the green circle.} \label{circHighFig}
\end{figure}

\begin{figure}[ht]
	\includegraphics*[width=0.96\linewidth,clip]{circularTooHighEnergy}
	\caption{A plot of energy against time for the circular orbit shown in figure \ref{circHighFig}. Gravitational energy is shown by the blue line, kinetic energy by the orange line, and total energy by the green line in between.} \label{circHighEn}
\end{figure}

An elliptical orbit was also modelled for the same time period and initial height, but with an initial velocity of 7500ms$^{-1}$, shown in figure \ref{elDefFig}. Again, this orbit was stable, completing several orbits in the time period, and this is also confirmed by figure \ref{elDefEn}, which shows the energy of the system over time. Several clear periods can be seen, and while the gravitational and kinetic energies oscillate, the total energy remains constant. The periodic nature of the energy is caused by the rocket coming closer and moving away from the Earth, and is a direct consequence of Kepler's Second Law \cite{Kepler1992}. The magnitude of the kinetic and gravitational energy is highest when the rocket is closest to the Earth, and the least when at apoapsis.

\begin{figure}[ht]
	\includegraphics*[width=0.96\linewidth,clip]{ellipticalDefault}
	\caption{Position of the orbit of a rocket with initial height of 3621km above the Earth's surface and initial velocity 7500ms$^{-1}$. The path of the rocket is shown in red, the red dot represents the current position of the rocket, and the Earth is represented by the green circle.} \label{elDefFig}
\end{figure}

\begin{figure}[ht]
	\includegraphics*[width=0.96\linewidth,clip]{ellipticalDefaultEnergy}
	\caption{A plot of energy against time for the elliptical orbit shown in figure \ref{elDefFig}. Gravitational energy is shown by the blue line, kinetic energy by the orange line, and total energy by the green line in between.} \label{elDefEn}
\end{figure}


For the slingshot around the moon, there were two options for a stable orbit; a figure-of-eight passing between the Earth and Moon twice, or an elliptical orbit, going round the outside of the Moon without crossing its previous path. The figure-of-eight was easier to implement, as a larger range of values were able to give a path that did not crash into the Moon, however a lot of trial and error was needed to find said values. Figure \ref{moonFoEFig} shows an example of such an orbit, with an initial height of 629km above the Earth's surface, and velocity 10559.9ms$^{-1}$. It was easiest to launch the rocket when it was in Low Earth orbit from the side of the Earth that was opposite the Moon. For the elliptical orbit around the Moon, an initial velocity of 10591ms$^{-1}$ was used. Figure \ref{moonEFig} shows the path taken for this orbit. It was quite surprising to see that a difference of only 31.1ms$^{-1}$ in the initial velocity was all that was needed to obtain each type of orbit, and upon further investigation, it was found that changing the initial velocity in the order of $\times$10$^{-1}$ decided whether the rocket crashed into the moon or not. The figure-of-eight orbit had a smaller period than the elliptical one, completing one orbit in about 850000 seconds, or 9 days, 20 hours and 6 minutes. The elliptical orbit took about 1050000 seconds to complete one period, or 12 days, 3 hours and 36 minutes. The magnitude of the gravitational and kinetic energy of the rocket on the elliptical path was lower than that of the figure-of-eight path, and this can be demonstrated by the smaller second peaks in figure \ref{moonEEn} than in figure \ref{moonFoEEn}, likely caused by the larger distance between the Earth and the rocket in the elliptical path. This is likely caused by the larger loop (seen at the top of figure \ref{moonEFig}) causing the rocket to enter orbit around the Earth again on a wider path. The rocket on the figure-of-eight path does not have this issue, and passes much closer to the Moon and Earth, evidenced by a small peak at roughly 400000 seconds in figure \ref{moonFoEEn} where it passes the Moon, and the large second peaks when it returns to the Earth.

\begin{figure}[ht]
	\includegraphics*[width=0.96\linewidth,clip]{moonFoE}
	\caption{Lunar slingshot orbit of a rocket with initial height of 629km above the Earth's surface and initial velocity 10559.9ms$^{-1}$. The path of the rocket is shown in red, the red dot represents the current position of the rocket, the Moon is represented by the grey dot, and the Earth is represented by the green circle. Both axes are of order 10$\times10^{8}$.} \label{moonFoEFig}
\end{figure}

\begin{figure}[ht]
	\includegraphics*[width=0.96\linewidth,clip]{moonFoEEnergy}
	\caption{A plot of energy against time for the lunar orbit shown in figure \ref{moonFoEFig}. Gravitational energy is shown by the blue line, kinetic energy by the orange line, and total energy by the green line in between.} \label{moonFoEEn}
\end{figure}

\begin{figure}[ht]
	\includegraphics*[width=0.96\linewidth,clip]{moonE}
	\caption{Lunar slingshot orbit of a rocket with initial height of 629km above the Earth's surface and initial velocity 10591ms$^{-1}$. The path of the rocket is shown in red, the red dot represents the current position of the rocket, the Moon is represented by the grey dot, and the Earth is represented by the green circle. Both axes are of order 10$\times10^{8}$.} \label{moonEFig}
\end{figure}

\begin{figure}[ht]
	\includegraphics*[width=0.96\linewidth,clip]{moonEEnergy}
	\caption{A plot of energy against time for the lunar orbit shown in figure \ref{moonEFig}. Gravitational energy is shown by the blue line, kinetic energy by the orange line, and total energy by the green line in between.} \label{moonEEn}
\end{figure}




\section{IMPROVEMENTS TO CODE}

As can be seen in the energy graphs (for example figure \ref{circDefEn}), at the end of the simulation, all the energy values jump to 0. This may be because there is an extra time value included in the array, where none exists in the energy arrays, but this could be fixed to remove this, by using validation checks to remove extra zeros at the end of the array. As well as this, the animation used for the graphs crashes when the end of the array is reached. I spent several hours trying to debug this, but could not understand why an extra index was being used for $N$, the total number of iterations, when this was the same as the size of the $x$ and $y$ arrays being passed to the function controlling the animation. An improvement would be to ensure that the animation function was being passed the correct number of iterations. Another improvement to the code would be to make the size of the Earth and Moon on the graphs to scale, as often when the rocket crashes, it is displayed as being in space, when in actuality it has hit the Moon/Earth, and this isn't displayed very well. To improve the simulation, the equation of motion could be adapted to account for relativistic gravitational effects, however as this is outside my area of knowledge I did not feel comfortable implementing this. One final improvement that could be made would be to move a large amount of the code into more functions, as several sections are reused very often. This is not difficult, and was only restricted by time, but an example of a useful function to help remove repetition would be implementing a function to handle all the Pythagoras, which would make the code easier to read.


\section{CONCLUSIONS}

%What have we learnt from this experiment?  How might we do things better?
The 4th-order Runge-Kutta method has been shown to be invaluable in numerically evaluating ordinary differential equations and modelling a satellite in orbit using Newton's gravity. The figure-of-eight Moon orbit is easier to perform, with a larger tolerance in the initial height and velocity values being able to produce a successful orbit. The code could have been improved to include gravitational effects from general relativity.

\section{REFERENCES}
%Do not include material in your reference list that is not specifically referred to in the text.  It is not a list of
%things you have read but rather a list of things you refer to in the text. Only refer to web based material if
%absolutely necessary. You should refer, where ever possible, to the original material in a textbook or academic
%journal.   A good report should have many references - showing that the subject has been widely researched.  The style
%to be used for the references is shown below.

\bibliographystyle{report}
\bibliography{C:/Users/Francis/Documents/Bibtex/Coding-Report-4}



\end{document}
